Commmonsense Question Answering (CQA) focuses on developing systems that can answer questions require a deep understanding of common sense knowledge and human-like reasoning. Unlike conventional question-answering systems, where answers are derived from explicit information, CQA requires the system to reason from implicit knowledge and real-world situations.

One of the most important data sources for CQA is the Commonsense Question Answering dataset. This consists of approximately 12,000 multiple choice questions, one correct and four noisy answers, built on ConceptNet. 

Additionally, the Commonsense Explanations (CoS-E) dataset adds explanations to the CQA dataset. The CoS-E dataset provides two types of explanations:
\begin{itemize}
    \item Selected explanations: Text fragments in the question are highlighted to explain why the answer was selected.
    \item Open-ended explanations: Free explanations written in natural language.
\end{itemize}

A prominent method in this area is the Commonsense Auto-Generated Explanation (CAGE) model. This model uses human-generated explanations to fine-tune the language model, helping it automatically generate useful explanations for answers.
